\documentclass{article}
\usepackage{amsmath}

\title{WorkoutSpot Algorithms}
\author{Leptos}
\date{2021-01-05}

\begin{document}
\maketitle

This article intends to serve as a reference for the formulae used in 
WorkoutSpot. The code (in WorkoutSpot) should have links to the original
formulae (as does this document), however referenced formulae may have 
inconsistent variables, or components of a formula may be scattered.

\section{Notation and Values}

This document uses the following symbols and notation.

altitude (height): $h$

latitude: $\phi$

longitude: $\lambda$

semi-major axis: $a$ ($6378137$ meters is used for Earth)
\cite[Fixed radius]{wikipedia_Earth_radius}

semi-minor axis: $b$ ($6356752$ meters is used for Earth)
\cite[Fixed radius]{wikipedia_Earth_radius}

\pagebreak
\section{Distances between Earth Coordinates}

\begin{verbatim}
    -[WSCoordinateAnalysis stepSpace]  
\end{verbatim}

This method uses a modified Haversine formula to find the
distance $d$ between two coordinates on Earth. The radius $r$ of Earth
is computed at each of the input latitudes $\phi$ using a geocentric
radius formula. 

Radius formula \cite[Geocentric radius]{wikipedia_Earth_radius}:
\begin{equation}
    r = \sqrt{\frac{
        \left(a^2 \cos \left(\phi\right)\right)^2 + 
        \left(b^2 \sin \left(\phi\right)\right)^2
    }{
        \left(a \cos \left(\phi\right)\right)^2 + 
        \left(b \sin \left(\phi\right)\right)^2
    }}
\end{equation}

To decrease the number of vector operations 
(i.e., operations that must be performed on non-constant values),
distribute the exponents. This allows $\sin{\phi}$ and $\cos{\phi}$
to be squared once, then used in both the numerator, and denominator.
\begin{equation}
    r = \sqrt{\frac{
        a^4 \cos^2 \left(\phi \right)+ b^4 \sin^2 \left(\phi\right)
    }{
        a^2 \cos^2 \left(\phi\right) + b^2 \sin^2 \left(\phi\right)
    }}
\end{equation}

Haversine formula \cite[Formulation]{wikipedia_Haversine_formula}:
\begin{equation}
    d = 2r \arcsin \left(\sqrt{
        \sin^2 \left(\frac{\phi_2 - \phi_1}{2}\right) + 
        \cos \left(\phi_1\right) \cos \left(\phi_2\right)
        \sin^2 \left(\frac{\lambda_2 - \lambda_1}{2}\right)
    }\right)
\end{equation}


I modified the formula to use the radii of each location, instead of
a single radius; the modification is straight forward: 
$2r \rightarrow \left(r_1 + r_2\right)$

Modified haversine formula:
\begin{equation}
    d = \left(r_1 + r_2\right) \arcsin \left(\sqrt{
        \sin^2 \left(\frac{\phi_2 - \phi_1}{2}\right) + 
        \cos \left(\phi_1\right) \cos \left(\phi_2\right)
        \sin^2 \left(\frac{\lambda_2 - \lambda_1}{2}\right)
    }\right)
\end{equation}

\pagebreak
\section{Earth Coordinates to XYZ}

\begin{verbatim}
    -[WSCoordinateAnalysis globeMapForAltitudes:]  
\end{verbatim}

This method uses a geodetic to ECEF formula 
\cite[From geodetic to ECEF coordinates]{wikipedia_Geographic_coordinate_conversion} 
to convert latitude $\phi$, longitude $\lambda$, altitude $h$ 
to XYZ coordinates, 
where $\left(0, 0, 0\right)$ is the center of Earth.

\begin{align}
    e^2 &= 1 - \frac{b^2}{a^2} \\
    N(\phi) &= \frac{a}{\sqrt{1 - e^2 \sin^2 \left(\phi\right) }} \\
    X &= \left(N\left(\phi\right) + h\right)
        \cos\left(\phi\right) \cos\left(\lambda\right) \\
    Y &= \left(N\left(\phi\right) + h\right)
        \cos\left(\phi\right) \sin\left(\lambda\right) \\
    Z &= \left(\frac{b^2}{a^2} N\left(\phi\right) + h\right)
        \sin\left(\phi\right)
\end{align}

\pagebreak
\section{Geodetic Parameterization}

Linear Scalar Parameterization:
\begin{equation}
    \label{linear_scalar_parameterization}
    x = x_1 + \left(x_2 - x_1\right)t
\end{equation}

For the below equations assume there is an 
initial point $P_1\left(\phi_1, \lambda_1\right)$,
and final point $P_2\left(\phi_2, \lambda_2\right)$.
This will involve the point $P_0\left(\phi_0, \lambda_0\right)$,
where the great circle of $P_1$ and $P_2$ intersects the equator.
Lastly, the point $P\left(\phi, \lambda\right)$ will be computed.

Initial ($P_1$) Course, $\alpha_1$ 
\cite[Course]{wikipedia_Great-circle_navigation}:
\begin{equation}
    \tan{\alpha_1} = \frac{
        \cos{\phi_2} \sin{\left(\lambda_2 - \lambda_1\right)}
    }{
        \cos{\phi_1} \sin{\phi_2} - 
            \sin{\phi_1}\cos{\phi_2}
                \cos{\left(\lambda_2 - \lambda_1\right)}
    }
\end{equation}

Final ($P_2$) Course, $\alpha_2$
\cite[Course]{wikipedia_Great-circle_navigation}:
\begin{equation}
    \tan{\alpha_2} = \frac{
        \cos{\phi_2} \sin{\left(\lambda_2 - \lambda_1\right)}
    }{
        -\cos{\phi_2} \sin{\phi_1} + 
            \sin{\phi_2}\cos{\phi_1}
                \cos{\left(\lambda_2 - \lambda_1\right)}
    }
\end{equation}

Equator ($P_0$) Course, $\alpha_0$
\cite[Finding way-points]{wikipedia_Great-circle_navigation}:
\begin{equation}
    \sin{\alpha_0} = \sin{\alpha_1} \cos{\phi_1}
\end{equation}

Equator ($P_0$) Course (with increased accuracy when the
great circle nears the poles), $\alpha_0$
\cite[Finding way-points]{wikipedia_Great-circle_navigation}:
\begin{equation}
    \tan{\alpha_0} = \frac{
        \sin{\alpha_1} \cos{\phi_1}
    }{
        \sqrt{\cos^2 {\alpha_1} + \sin^2 {\alpha_1} \sin^2 {\phi_1}}
    }
\end{equation}

Angular Distance (between $P_0$ and $P_1$), $\sigma_{01}$
\cite[Finding way-points]{wikipedia_Great-circle_navigation}:
\begin{equation}
    \tan{\sigma_{01}} = \frac{
        \tan{\phi_1}
    }{
        \cos{\alpha_1}
    }
\end{equation}

Angular Distance (between $P_0$ and $P_2$), $\sigma_{02}$
\cite[Finding way-points]{wikipedia_Great-circle_navigation}:
\begin{equation}
    \tan{\sigma_{02}} = \frac{
        \tan{\phi_2}
    }{
        \cos{\alpha_2}
    }
\end{equation}

Equator ($P_0$) Longitude, $\lambda_0$
\cite[Finding way-points]{wikipedia_Great-circle_navigation}:
\begin{equation}
    \tan{\left(\lambda_0 - \lambda_1\right)} = \frac{
        \sin{\alpha_0} \sin{\sigma_{01}}
    }{
        \cos{\sigma_{01}}
    }
\end{equation}

Select an angular distance, $\sigma$; we'll be using the equation
from \eqref{linear_scalar_parameterization} to select a value
between the angular distances from above.
\begin{equation}
    \sigma = \sigma_1 + \left(\sigma_2 - \sigma_1\right)t
\end{equation}

Computed ($P_0$) Latitude, $\phi$
\cite[Finding way-points]{wikipedia_Great-circle_navigation}:
\begin{equation}
    \sin{\phi} = \cos{\alpha_0} \sin{\sigma}
\end{equation}

Computed ($P_0$) Latitude (with increased accuracy around the poles), 
$\phi$ \cite[Finding way-points]{wikipedia_Great-circle_navigation}:
\begin{equation}
    \tan{\phi} = \frac{
        \cos{\alpha_0} \sin{\sigma}
    }{
        \sqrt{\cos^2{\sigma} + \sin^2{\alpha_0} \sin^2{\sigma}}
    }
\end{equation}

Computed ($P_0$) Longitude, $\lambda$
\cite[Finding way-points]{wikipedia_Great-circle_navigation}:
\begin{equation}
    \tan{\left(\lambda - \lambda_0\right)} = \frac{
        \sin{\alpha_0} \sin{\sigma}
    }{
        \cos{\sigma}
    }
\end{equation}


\begin{thebibliography}{9}

    \bibitem{wikipedia_Earth_radius}
\begin{verbatim*}
https://en.wikipedia.org/wiki/Earth_radius
\end{verbatim*}
    
    \bibitem{wikipedia_Haversine_formula}
\begin{verbatim*}
https://en.wikipedia.org/wiki/Haversine_formula
\end{verbatim*}
        
    \bibitem{wikipedia_Geographic_coordinate_conversion}
\begin{verbatim*}
https://en.wikipedia.org/wiki/Geographic_coordinate_conversion
\end{verbatim*}
    
    \bibitem{wikipedia_Great-circle_navigation}
\begin{verbatim*}
https://en.wikipedia.org/wiki/Great-circle_navigation
\end{verbatim*}

\end{thebibliography}

\end{document}
